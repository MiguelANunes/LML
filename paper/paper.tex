%%
%% This is file `sample-sigconf.tex',
%% generated with the docstrip utility.
%%
%% The original source files were:
%%
%% samples.dtx  (with options: `sigconf')
%% 
%% IMPORTANT NOTICE:
%% 
%% For the copyright see the source file.
%% 
%% Any modified versions of this file must be renamed
%% with new filenames distinct from sample-sigconf.tex.
%% 
%% For distribution of the original source see the terms
%% for copying and modification in the file samples.dtx.
%% 
%% This generated file may be distributed as long as the
%% original source files, as listed above, are part of the
%% same distribution. (The sources need not necessarily be
%% in the same archive or directory.)
%%
%%
%% Commands for TeXCount
%TC:macro \cite [option:text,text]
%TC:macro \citep [option:text,text]
%TC:macro \citet [option:text,text]
%TC:envir table 0 1
%TC:envir table* 0 1
%TC:envir tabular [ignore] word
%TC:envir displaymath 0 word
%TC:envir math 0 word
%TC:envir comment 0 0
%%
%%
%% The first command in your LaTeX source must be the \documentclass command.
\documentclass[sigconf]{acmart}

%%
%% \BibTeX command to typeset BibTeX logo in the docs
\AtBeginDocument{%
  \providecommand\BibTeX{{%
    Bib\TeX}}}

%% Rights management information.  This information is sent to you
%% when you complete the rights form.  These commands have SAMPLE
%% values in them; it is your responsibility as an author to replace
%% the commands and values with those provided to you when you
%% complete the rights form.
\copyrightyear{2022}
\acmYear{2022}
\setcopyright{acmcopyright}\acmConference[SBLP 2022]{XXVI Brazilian Symposium
on Programming Languages}{October 6--7, 2022}{Virtual Event, Brazil}
\acmBooktitle{XXVI Brazilian Symposium on Programming Languages (SBLP 2022),
October 6--7, 2022, Virtual Event, Brazil}
\acmPrice{15.00}
\acmDOI{10.1145/3561320.3561329}
\acmISBN{978-1-4503-9744-5/22/10}


%%
%% Submission ID.
%% Use this when submitting an article to a sponsored event. You'll
%% receive a unique submission ID from the organizers
%% of the event, and this ID should be used as the parameter to this command.
%%\acmSubmissionID{123-A56-BU3}

%%
%% For managing citations, it is recommended to use bibliography
%% files in BibTeX format.
%%
%% You can then either use BibTeX with the ACM-Reference-Format style,
%% or BibLaTeX with the acmnumeric or acmauthoryear sytles, that include
%% support for advanced citation of software artefact from the
%% biblatex-software package, also separately available on CTAN.
%%
%% Look at the sample-*-biblatex.tex files for templates showcasing
%% the biblatex styles.
%%

%%
%% The majority of ACM publications use numbered citations and
%% references.  The command \citestyle{authoryear} switches to the
%% "author year" style.
%%
%% If you are preparing content for an event
%% sponsored by ACM SIGGRAPH, you must use the "author year" style of
%% citations and references.
%% Uncommenting
%% the next command will enable that style.
%%\citestyle{acmauthoryear}

\usepackage{minted}
\usepackage{bussproofs}
\usepackage{multirow}

%%
%% end of the preamble, start of the body of the document source.
\begin{document}
%%
%% The "title" command has an optional parameter,
%% allowing the author to define a "short title" to be used in page headers.
\title{A Sound Deep Embedding of Arbitrary Normal Modal Logics in Coq}

%%
%% The "author" command and its associated commands are used to define
%% the authors and their affiliations.
%% Of note is the shared affiliation of the first two authors, and the
%% "authornote" and "authornotemark" commands
%% used to denote shared contribution to the research.

\author{Ariel Agne da Silveira}
\affiliation{\institution{Federal University of Ouro Preto}\country{}}
\email{ariel.silveira@aluno.ufop.edu.br}

\author{Miguel Nunes}
\affiliation{\institution{Santa Catarina State University}\country{}}
\email{miguel.alfredo.nunes@gmail.com}

\author{Karina Roggia}
\affiliation{\institution{Santa Catarina State University}\country{}}
\email{karina.roggia@udesc.br}

\author{Paulo Torrens}
\affiliation{\institution{Santa Catarina State University}\country{}}
\email{paulotorrens@gnu.org}

\author{Rodrigo Ribeiro}
\affiliation{\institution{Federal University of Ouro Preto}\country{}}
\email{rodrigo.ribeiro@ufop.edu.br}

%%
%% By default, the full list of authors will be used in the page
%% headers. Often, this list is too long, and will overlap
%% other information printed in the page headers. This command allows
%% the author to define a more concise list
%% of authors' names for this purpose.
\renewcommand{\shortauthors}{da Silveira et al.}

%%
%% The abstract is a short summary of the work to be presented in the
%% article.
\begin{abstract}
  This work describes an encoding of modal logics using the Coq proof assistant.
  Our formalization differs from previous attempts by providing a uniform
  representation of several systems for modal logic using Coq's rich type
  structure. We illustrate the usefulness of our library in a formalization of
  L\"ob's theorem which closely follows a classical proof of this result.
\end{abstract}

\begin{CCSXML}
  <ccs2012>
    <concept>
        <concept_id>10003752.10003790.10003793</concept_id>
        <concept_desc>Theory of computation~Modal and temporal logics</concept_desc>
        <concept_significance>500</concept_significance>
        </concept>
    <concept>
        <concept_id>10003752.10003790.10002990</concept_id>
        <concept_desc>Theory of computation~Logic and verification</concept_desc>
        <concept_significance>500</concept_significance>
        </concept>
  </ccs2012>
\end{CCSXML}

\ccsdesc[500]{Theory of computation~Modal and temporal logics}
\ccsdesc[500]{Theory of computation~Logic and verification}

%%
%% Keywords. The author(s) should pick words that accurately describe
%% the work being presented. Separate the keywords with commas.
\keywords{Modal logic, soundness, Coq proof assistant}

\maketitle

\section{Introduction}\label{sec:intro}


The start of modal logics can be traced back to Lewis and his definition of 
logical systems with strict implication instead of material implication \cite{Lewis1918}.
Based in this work, independently of classical propositional logic,
defined the systems $\mathsf{S1}$--$\mathsf{S5}$~\cite{Lewis1932}, where $\mathsf{S4}$ and $\mathsf{S5}$
are well known normal modal systems. It was, in fact, G{\"o}del that had the idea 
to build a modal logic system as an extension of the classical propositional calculus, 
when constructing a system similar to $\mathsf{S4}$ in \cite{Godel1933}.

Despite that the original motivation of modal systems was to deal with strict implication, 
for many years modal logic was seen as ``the logic of necessity and possibility''. 
Only after the introduction of Kripke semantics they were also seen as ``simple yet
expressive languages for talking about relational structures''~\cite{Blackburn-ModalLogics}. 
A great multitude of modal systems, with various kinds of modal operators, each of them with 
different meanings or properties, have then been constructed in order to provide effective
formalisms for talking about time, space, knowledge, beliefs, actions, 
obligations, among others.

The interpretation of modal logic simply as ``the logic of necessity'' restricts 
not only its applications but also the range of modal systems. In fact, modal logic 
can also be seen as ``the study of \emph{modalities} -- logical operations that qualify 
assertions about the truth of statements''~\cite{Goldblatt-MathofModality}. In this sense, 
the necessitation rule (from $\varphi$ infer $\Box\varphi$) may be relaxed. This brings 
a number of modal systems known as non-normal, whose semantics were defined only in the 
1970's by Montague~\cite{Montague1970} and Scott~\cite{Scott_AdviceModal1970}.

Such wide range of highly expressive logical languages is the ideal tool to 
formalize the many complexities of modern advanced software systems, 
such as those found in cars, airplanes, medical devices and banking applications.
In combining said modal logics with interactive proof assistants,
such as Coq~\cite{manual_coq} and Isabelle/HOL~\cite{manual_isabelle}, both of
which have been applied with success in non-trivial projects like a certified compiler 
for C programming language~\cite{Leroy09} and the kernel of seL4 operating
system~\cite{Klein10}, it is possible to formalize and verify highly complex computer
systems.

Although there are some dedicated provers for modal logics~\cite{Gleissner17,Mora11},
we believe that using mature tools, like a proof assistant, is a better option for
verifying facts using modal logics. We justify this choice by: 1) the recent
successful histories on verification of software and
mathematics~\cite{Leroy09,Klein10,Gonthier13,Gonthier08}; 2) there is a growing
community of developers using such tools thanks to 
textbooks on the subject~\cite{Nipkow14,Bertot10,Chlipala13} which allows
newcomers to join the game of formal verification; and 3) the custom automation
facilities supported by current proof assistants allows for trivial facts to
be proven using decision procedures, relieving the user from writing proof
scripts for formulas that can be checked automatically.

In this context, the present work aims to contribute by constructing a library
that formalizes a family of normal modal logics and their related properties in Coq.
By using such a library, a developer may use all of Coq's resources to prove
modal logics facts.


% Our society depends heavily on computing technology: our cars, airplanes, medical devices
% and banking applications all rely on software artifacts. In this scenario,
% \emph{formal methods} play a crucial role to ensure that a program follows its
% specification, thus avoiding failures which can cause castrophic outcomes. Interactive
% proof assistants, like Coq~\cite{manual_coq} and Isabelle/HOL~\cite{manual_isabelle}, can
% be used to verify software components and have been applied with success in
% non-trivial projects like a certified compiler for C programming
% language~\cite{Leroy09} and the kernel of seL4 operating system~\cite{Klein10}.

% Usually, complex software systems are formed by various inter-communicating
% components. A key aspect of the design of such programs is how to ensure that
% a failure of a single component does not compromise the whole system.
% The main difficulty is such formalization tasks is how to guarantee correct
% error detection and that its corresponding fix will ensure that the software
% still continue to work under abnormal conditions. One way to express such
% requirements is using \emph{modal logics}, which can specify communication
% patterns and knowledge in software agents~\cite{Huth_Ryan}.

% While modal logics allow the modelling of such requirements, proving properties
% about them may be a hard task without proper tool support. There are some
% dedicated provers for modal logics~\cite{Gleissner17,Mora11}, but we believe
% that using mature tools, like a proof assistant, is a better option for
% verifying facts using modal logics. We justify this choice by: 1) the recent
% successful histories on verification of software and
% mathematics~\cite{Leroy09,Klein10,Gonthier13,Gonthier08}; 2) There is a growing
% community of developers using such tools thanks to 
% textbooks on the subject~\cite{Nipkow14,Bertot10,Chlipala13} which allows
% newbies to join the game of formal verification and 3) The custom automation
% facilities supported by current proof assistants allows that trivial facts can
% be proved using decision procedures, relieving the user from writing proof
% scripts for formulas that can be checked automatically.

% In this context, the present work aims to contribute by constructing a library
% that formalize a family of normal modal logics and its related properties in Coq
% by parametrizing the proofs over a set of possible axioms. 
% In using such a library, a developer could use all Coq's resources to prove
% modal logics facts.

More specifically, we contribute:

\begin{itemize}
  \item We define a \emph{deep embedding} of modal logic syntax in Coq and
    define its Kripke semantics.
  \item We model the Hilbert deductive system as a Coq inductive type that is
    parameterized by different axiom systems like K, D, B, T, K4, K5, S4, S5 and GL,
    and prove its soundness with respect to Kripke semantics.
    In such, we also formalize several frame correspondence results.
  \item As an example of usage, we show a proof of L\"ob's theorem using the deductive system
    developed in our library.
\end{itemize}

The rest of this work is organized as follows: Section~\ref{sec:background} provides
brief introduction on modal logics and the Coq proof assistant. Details about
the implementation of our Coq library are presented in
Section~\ref{sec:implementation}. A case study in proving L\"ob's theorem is
the subject of Section~\ref{sec:case}. Related works are discussed
on Section~\ref{sec:related} and Section~\ref{sec:conclusion} draws some conclusions and
provide some pointers for future work.

All source code in this article has been formalized in Coq version 8.15 and
is available at \url{https://github.com/funcao/ModalLibrary/} under a permissive free software license.

\section{Background}\label{sec:background}

\paragraph{An overview of Coq proof assistant.}

Coq is a proof assistant based on the calculus of inductive
constructions (CIC)~\cite{manual_cic}, a higher-order typed
$\lambda$-calculus extended with inductive definitions.  Theorem
proving in Coq follows the ideas of the so-called
``BHK-in\-ter\-pre\-ta\-tion''\footnote{Abbreviation of Brouwer, Heyting,
  Kolmogorov, de Bruijn and Martin-L\"of Correspondence. This is also
  known as the Curry-Howard isomorphism.}, where types represent
logical formulas, $\lambda$-terms represent proofs
\cite{Bertot10} and the task of checking if a piece of text is a
proof of a given formula corresponds to checking if the term that
represents the proof has the type corresponding to the given formula.

In order to make writing proofs easier, Coq provides
\emph{tactics}, which are commands that can be used to construct proof
terms in a more user-friendly way. As an example, consider the task of proving the following 
formula of the propositional logic:
\[
(A \to B)\to (B\to C) \to A \to C
\]
In Coq, such theorem can be expressed as:
\begin{minted}{coq}
Section EXAMPLE.
  Variables A B C: Prop.
  Theorem example: (A -> B) -> (B -> C) -> A -> C.
  Proof.
    intros H H' HA.
    apply H'.
    apply H.
    assumption. 
  Qed.
End EXAMPLE.
\end{minted}
In the above, we have defined a Coq section named
\mintinline{coq}{EXAMPLE}\footnote{In Coq, we can use sections to delimit the
  scope of local variables.} which declares variables \mintinline{coq}{A},
\mintinline{coq}{B} and \mintinline{coq}{C} as being propositions (i.e. with type
\mintinline{coq}{Prop}). Tactic \mintinline{coq}{intros} introduces variables
\mintinline{coq}{H}, \mintinline{coq}{H'} and \mintinline{coq}{HA} into the typing context,
respectively with types \mintinline{coq}{A -> B}, \mintinline{coq}{B -> C}, and
\mintinline{coq}{A}, and leaves goal \mintinline{coq}{C} to be proved. Tactic
\mintinline{coq}{apply}, used with a term \mintinline{coq}{t}, generates goal
\mintinline{coq}{P}
when there exists \mintinline{coq}{t: P -> Q} in the typing context and the
current goal is \mintinline{coq}{Q}. Thus, \mintinline{coq}{apply H'} changes the goal
from \mintinline{coq}{C} to \mintinline{coq}{B} and \mintinline{coq}{apply H}
changes the goal to \mintinline{coq}{A}. Tactic \mintinline{coq}{assumption}
traverses the typing context to find a hypothesis that matches the goal,
concluding the proof.

We define next a proof of the previous propositional logical formula
that, in contrast to the previous proof, that was built using tactics, is coded
directly as a function:
\begin{minted}{coq}
Definition example: (A -> B) -> (B -> C) -> A -> C :=
  fun (H: A -> B) (H': B -> C) (HA: A) => H' (H HA).
\end{minted}
However, even for very simple theorems, coding a definition directly
as a Coq term can be a hard task. Because of this, the use of tactics
has become the standard way of proving theorems in Coq. Furthermore,
the Coq proof assistant provides not only a great number of tactics
but also a domain-specific language for scripted proof automation,
called $\mathcal{L}$tac. Details about $\mathcal{L}$tac and Coq can be found
in~\cite{Chlipala13,Bertot10,manual_coq}.

\paragraph{An overview of modal logic.} Modal logics are designed for reasoning
about truth across various --- abstract --- worlds. In such logics, a
proposition may be true in some world, but false in another one. The systems of
modal logic considered in this work extend traditional propositional logic
with two operators on propositions: $\Box$ (box) and $\Diamond$ (diamond).
The syntax for modal logic formulas is defined by the following context free
grammar (where
$\mathcal{P} = \{A,B,C,...\}$ is the countable set of propositional variables):
\[
  \begin{array}{lcl}
    \varphi & ::= & \mathcal{P}\,|\,\neg\varphi\,|\,\varphi\land\varphi\,|\,\varphi\lor\varphi\,|\,\varphi\to\varphi\,|\,\Box\varphi\,|\,\Diamond\varphi
  \end{array}  
\]
Intuitively, the semantics of the operator $\Box$ is similar to an universal
quantifier, since it expresses \emph{necessity}: $\Box\,\varphi$ is true in the
current world iff $\varphi$ is true in every world accessible from the current
one. Similarly, the semantics of operator
$\Diamond$ is that a formula $\Diamond\varphi$ is true in the current
world iff $\varphi$ is \emph{possible}, i.e. true in at least some world
accessible from the current world. We the remaining of the paper, we also write
$\phi_1\leftrightarrow\phi_2$ to mean $\phi_1\to\phi_2\land\phi_2\to\phi_1$.

Formally, we can only interpret a formula with respect to a \emph{model}, which is
defined as follows.
A \emph{frame} $\mathcal{F} = \langle W, R \rangle$ consists of a pair formed by
a non-empty set of worlds, $W$, and a binary accessibility relation between
worlds, $R \subseteq W \times W$. A pair $w_1Rw_2$ denotes that world $w_2$ is
accessible from $w_1$. A model $\mathcal{M} = \langle \mathcal{F}, \mathcal{V} \rangle$
is formed by a frame $\mathcal{F}$ and a total labeling function
$\mathcal{V} : \mathcal{P}\to 2^W$, which indicates the set of worlds where each propositional variable holds, i.e. when $w \in \mathcal{V}(A)$
for some $A \in\mathcal{P}$, we say that $A$ holds in $w$. The satisfiability of a formula
in a given world is defined by inductive relation $\mathcal{M}\,;\,w\models \varphi$ as
follows (notation $\mathcal{M}\,;\,w\not\models \varphi$ denotes
that $\mathcal{M}\,;\,w\models \varphi$ does not hold):

\begin{figure}[H]
\begin{enumerate}
  % \item $\mathcal{M}\,;\,w\not\models \bot$ and $\mathcal{M}\,;\,w\models \top$;
  \item $\mathcal{M}\,;\,w\models A$ iff $w \in \mathcal{V}(A)$;
  \item $\mathcal{M}\,;\,w\models \neg\varphi$ iff $\mathcal{M}\,;\,w\not\models\varphi$;
  \item $\mathcal{M}\,;\,w\models \varphi_1 \land \varphi_2$ iff
    $\mathcal{M}\,;\,w\models\varphi_1$ and $\mathcal{M}\,;\,w\models\varphi_2$;
  \item $\mathcal{M}\,;\,w\models \varphi_1 \lor \varphi_2$ iff
    $\mathcal{M}\,;\,w\models\varphi_1$ or $\mathcal{M}\,;\,w\models\varphi_2$;
  \item $\mathcal{M}\,;\,w\models \varphi_1 \to \varphi_2$ iff
    $\mathcal{M}\,;\,w\not\models\varphi_1$ or $\mathcal{M}\,;\,w\models\varphi_2$;
  \item $\mathcal{M}\,;\,w\models\Box\varphi$ iff $\forall y \in W. w R y \to \mathcal{M}\,;\,y \models\varphi$;
  \item $\mathcal{M}\,;\,w\models\Diamond\varphi$ iff $\exists y \in W. w R y \land \mathcal{M}\,;\,y \models\varphi$;
  \end{enumerate}
  \caption{Kripke semantics}
  \label{fig:kripke}
\end{figure}
We say that a formula $\varphi$ is satisfiable in a model $\mathcal{M}$,
denoted by $\mathcal{M}\models \varphi$, if $\forall w \in W. M\,;w\,\models \varphi$ holds.
Similarly, $\varphi$ is valid in a frame $\mathcal{F}$, written $\mathcal{F}\models\varphi$, if
it holds on all models $\mathcal{M} = \langle \mathcal{F},\mathcal{V}\rangle$, i.e.
for any $\mathcal{M}$, we have that $\mathcal{M}\models\varphi$. A formula
is valid, written $\models \varphi$, if it is valid in any frame $\mathcal{F}$.

Using the semantics, we can define the concept of \emph{entailment}: given a set
of formulas $\Gamma$ and a formula $\varphi$, we say that $\Gamma$ entails
$\varphi$, written $\Gamma\models\varphi$, iff $\forall \mathcal{M} . (\forall \varphi' \in\Gamma .
\mathcal{M} \models \varphi') \to \mathcal{M} \models \varphi$.

In this work, we consider Hilbert deductive systems for modal logic.
Essentially, the considered system adds axioms for the modalities
to the set of axioms for classical propositional logic.
As examples of modalities axioms we mention \textbf{K}, which express that $\Box$ distributes
over implication, and possibility which states that $\Diamond$ distributes
over disjunction.
\[
  \begin{array}{ll}
    \textbf{K} & \Box(\varphi_1 \to \varphi_2) \to (\Box \varphi_1 \to \Box \varphi_2)\\
    \textbf{Possibility} & \Diamond (\varphi_1 \lor \varphi_2) \to
                           (\Diamond \varphi_1 \lor \Diamond \varphi_2)\\
  \end{array}
\]
Apart from the axioms, we also need \emph{modus ponens} and necessitation rules.

\begin{figure}[H]
\begin{tabular}{cc}
    \AxiomC{$\Gamma\vdash\varphi_1\to\varphi_2$}
    \AxiomC{$\Gamma\vdash\varphi_1$}
    \RightLabel{MP}
    \BinaryInfC{$\Gamma\vdash\varphi_2$}
    \DisplayProof
  &
    \AxiomC{$\Gamma\vdash\varphi$}
    \RightLabel{Nec}
    \UnaryInfC{$\Gamma\vdash\Box \varphi$}
    \DisplayProof
\end{tabular}
\end{figure}
The set of axioms and the two previous rules form a sound and complete Hilbert-style
deductive system for modal logics~\cite{blackburn}.

A central aspect in the study of modal logics is the so-called \emph{frame
  correspondence}, which is a relationship between the
accessibility relation and the truth of certain formulas in frames containing
it. As an example, consider a frame $\mathcal{F} = \langle W,R
\rangle$ in which the relation $R$ is reflexive: $\forall w \in W. wRw$, i.e.
every world is accessible from itself. In such frame when $\Box A$ holds in
a world $w$, $w$ itself is among the accessible worlds in which $A$ must be true.
In this way, if the accessibility relation $R$ is reflexive, then whatever world
$w$ and formula $\varphi$ we take, the formula $\Box \varphi \to \varphi$ will
be true there. Examples of properties that characterize certain formula schemas are
presented in the table below.

\begin{figure}[H]
\begin{tabular}{|c|c|}
  \hline
  Property of $R$                          & Formula schema \\ \hline
  reflexive: $\forall w.wRw$               & $\Box A \to A$ \\ 
  serial: $\forall w\exists w'. wRw'$      & $\Box A \to \Diamond A$ \\
  symmetric: $\forall w\,w'.wRw' \to w'Rw$ & $A \to \Box\Diamond A$ \\
  \hline
\end{tabular}
\end{figure}
There are other properties of interest for a relation $R$ like being
transitive, functional, euclidean, and so on. More about the correspondence
of frames and deductive systems properties can be found elsewhere~\cite{Chellas,blackburn}.

\section{Embedding Modal Logic in Coq}\label{sec:implementation}

In this section, we describe the implementation details of our modal logic
library in Coq. We start by describing the structure of formula syntax and the
representation of frames and models, needed to define the Kripke semantics.
Next, we define a modular Hilbert-style deductive system. The modularization is
obtained by parameterizing it by a set of axioms used. This allows us to
generalize results about derivations to several deductive systems, independently
of the axioms used. We conclude by discussing the soundness and the frame
correspondence proofs.

\paragraph{Syntax and semantics of modal logics}
To implement our library, we use a \emph{deep embedding} of
formulas, i.e. we represent formulas as the following Coq inductive datatype.
\begin{minted}{coq}
Inductive formula: Set :=
| Lit    : nat -> formula
| Neg    : formula -> formula
| Box    : formula -> formula
| Dia    : formula -> formula
| And    : formula -> formula -> formula
| Or     : formula -> formula -> formula
| Implies: formula -> formula -> formula.
\end{minted}
Most constructors of the type have an immediate meaning, since they represent
connectives of propositional logic. The constructor \mintinline{coq}{Lit}
is used to construct propositional variables; constructor \mintinline{coq}{Box}
denotes the modal operator $\Box$ and \mintinline{coq}{Dia}, the operator
$\Diamond$. Furthermore, the type \mintinline{coq}{theory} is defined as a list
of formulas.

\begin{minted}{coq}
Definition theory := list formula.
\end{minted}

Before defining the Kripke semantics, we need to represent the model used to
interpret a modal logic formula. We encode a frame by a record formed by
a type \mintinline{coq}{W: Set} to represent the set of worlds and an
accessibility relation denoted by a type \mintinline{coq}{R: W -> W -> Prop}.
\begin{minted}{coq}
Record Frame: Type := {
  W: Set;
  R: W -> W -> Prop;
}.
\end{minted}
A model is then denoted by another record, formed by a frame and a labeling function,
which assigns the truth value of a variable at a given world.
\begin{minted}{coq}
Record Model: Type := {
  F: Frame; 
  v: nat -> (W F) -> Prop; 
}.
\end{minted}
Note that the type of the labeling function \mintinline{coq}{v} ensures that it
refers to a world from the set of possible worlds in the frame
\mintinline{coq}{F}\footnote{We recall that \mintinline{coq}{W F} is Coq's syntax for
accessing field \mintinline{coq}{W} from the record \mintinline{coq}{F}.}. The encoding of the Kripke semantics is done by a Coq function
that implements the relation of Figure~\ref{fig:kripke}. We omit the code
for propositional logics connectives for brevity. The interesting parts are for
variables (constructor \mintinline{coq}{Lit}) which uses the labeling function
\mintinline{coq}{v} from the model \mintinline{coq}{M}. The meaning of
modalities $\Box$ and $\Diamond$ is represented by quantifying over worlds
using the accessibility relation $R$. 
\begin{minted}{coq}
Fixpoint formula_valuation (M: Model) (w: W (F M))
                           (phi: formula): Prop :=
  match phi with
  | Lit     x
     => v M x w (* should be inhabited, i.e., true *)
  | Box     psi
     => forall w': W (F M), R (F M) w w' ->
                            formula_valuation M w' psi
  | Dia     psi
     => exists w': W (F M), R (F M) w w' /\
                            formula_valuation M w' psi
  (* some code omitted *)
  end.
\end{minted}
In order to ease the task of writing formulas and its semantics properties, we
define Coq notations that mimic the standard way of writing them.

\paragraph{The Hilbert deductive system}
Basically, a Hilbert-style deductive system allows the use of four
possibilities in a proof: 1) the use of a hypothesis; 2) the use of an axiom;
3) the use of the modus ponens rule or 4) the use of the generalization (necessity)
rule. The inductive type \mintinline{coq}{deduction} encodes these possibilities
as its constructors. The \mintinline{coq}{deduction} type is parameterized by
the set of axioms that form a system for some modal logic. In this way, our definition can be
used by several systems alike.
\begin{minted}{coq}
Inductive deduction (A: axiom -> Prop):
                          theory -> formula -> Prop :=
  (* Premise. *)
  | Prem: forall (t: theory)
                 (f: formula)
                 (i: nat),
          (nth_error t i = Some f) -> deduction A t f
  (* Axiom. *)
  | Ax: forall (t: theory)
               (a: axiom)
               (f: formula),
        A a -> instantiate a = f -> deduction A t f
  (* Modus Ponens. *)
  | Mp: forall (t: theory)
               (f g: formula)
               (d1: deduction A t (Implies f g))
               (d2: deduction A t f),
        deduction A t g
  (* Generalization. *)
  | Nec: forall (t: theory)
                (f: formula)
                (d1: deduction A t f),
         deduction A t (Box f).
\end{minted}
The constructors \mintinline{coq}{Prem}, \mintinline{coq}{Mp} and \mintinline{coq}{Nec}
represent rules for hypothesis, modus ponens and generalization, respectively.
Before discussing the \mintinline{coq}{Ax} constructor, we need to discuss how axioms are
represented in our library. The inductive type \mintinline{coq}{axiom} specifies
the different axioms supported by our library.
\begin{minted}{coq}
Inductive axiom: Set :=
  | ax1  : formula -> formula -> axiom
  | ax2  : formula -> formula -> formula  -> axiom
  (* some code omitted *)
  | axK  : formula -> formula -> axiom
  | axPos: formula -> formula -> axiom
  | axT  : formula -> axiom
  | axB  : formula -> axiom
  | axK4 : formula -> axiom
  | axD  : formula -> axiom
  | axK5 : formula -> axiom.
\end{minted}
Each constructor of type \mintinline{coq}{axiom} represents an axiom scheme and
its type stores the subformulas which compose the axiom. The type has
17 constructors: the first 10 encode axioms for propositional logic
and, the last 7 are for different formulas which are taken as axioms by common
systems of modal logic. Axioms are built using the function \mintinline{coq}{instantiate},
which receives an axiom and builds a corresponding formula by pattern matching as
follows:
\begin{minted}{coq}
Definition instantiate (a: axiom): formula :=
  match a with
  | ax1    p   q
     => [! p -> (q -> p) !]
  | ax2    p   q   r
     => [! (p -> (q -> r)) -> ((p -> q) -> (p -> r)) !]
  | ax3    p   q
     => [! (~ q -> ~ p) -> (p -> q) !]
  | axK    p   q
     => [! [] (p -> q) -> ([] p -> [] q) !]
  | axPos  p   q
     => [! <> (p \/ q) -> (<> p \/ <> q) !]
  (* some code omitted *)
  end.
\end{minted}
Notation \mintinline{coq}{[! p !]} is used to write modal logic formulas using a
syntax closer to mathematical practice using Coq. This is necessary to avoid
parsing ambiguities, since some symbols are also used by Coq's syntax.
Modalities $\Box$ and $\Diamond$ are denoted by \mintinline{coq}{[]}
and \mintinline{coq}{<>}, respectively.

The \mintinline{coq}{Ax} constructor uses the \mintinline{coq}{instantiate}
function to create a formula instance from a given system,
represented as a predicate
\mintinline{coq}{A: axiom -> Prop} which parameterizes our deduction type. An
axiom \mintinline{coq}{a} is valid within a system \mintinline{coq}{A} whenever
the type \mintinline{coq}{A a} is inhabited, i.e. it is not empty. Next, we
describe how to represent different systems of modal logic.

\paragraph{Systems of modal logics} In order to use the deductive system to
check the provability of formulas, we need to specify the axioms that are
included in the system. Our approach is to define predicates
which denote each system. The simplest set of axioms is usually referred as
\textbf{K} and it includes the axiom $\Box(A \to B) \to (\Box A \to \Box B)$
and all the axioms for propositional logic. We represent system \textbf{K}
by the following inductive predicate:
\begin{minted}{coq}
Inductive K: axiom -> Prop :=
  | K_ax1: forall p q, K (ax1 p q)
  | K_ax2: forall p q r, K (ax2 p q r)
  (* some code omitted *)
  | K_axK: forall p q, K (axK p q)
\end{minted}
Type \mintinline{coq}{K} is a predicate over axioms: it has one constructor for
each of the valid axioms which form the system. Given a value
of \mintinline{coq}{axiom}, a value of type \mintinline{coq}{K a} denotes a
proof that a given axiom is part of the system \textbf{K}. This is the main
trick used by the constructor \mintinline{coq}{Ax}, of type
\mintinline{coq}{deduction}, to generically represent axioms for
different systems of modal logics uniformly. In order to understand how it
works, let us review its type (below, remember that \mintinline{coq}{A: axiom -> Prop}
denotes a system of axioms for modal logics):
\begin{minted}{coq}
Ax: forall (t: theory)
           (a: axiom)
           (f: formula),
           A a -> instantiate a = f -> deduction A t f
\end{minted}
Note that the fourth parameter of constructor \mintinline{coq}{Ax} is a value
of type \mintinline{coq}{A a}, which is only inhabited if the predicate
\mintinline{coq}{A} holds for the axiom \mintinline{coq}{a}. Through this
evidence, is possible to create the corresponding axiom formula instance by
using function \mintinline{coq}{instantiate}. In this way, we are able to
represent all systems of axioms for modal logics using a single type
for its axioms.

\paragraph{Properties} A key property of the Hilbert deductive system as formalized
in our library is its soundness (through system K). If a formula $\varphi$ is derived from a
context $\Gamma$, i.e. $\Gamma\vdash\varphi$, then $\varphi$ is entailed within the deep embedding
from the set of hypotheses $\Gamma$, i.e. $\Gamma\models\varphi$. The Coq statement for the soundness
result is given as follow:

\begin{minted}{coq}
Theorem soundness:
  forall G p, (K; G |-- p) -> (G ||= p).
\end{minted}

The proof proceeds by induction on the derivation of $\Gamma \vdash\varphi$ and
it has a case for each of the 11 axioms of the Hilbert system for propositional
logics, one for the $K$ axiom and cases for using hypothesis and the rules modus
ponens. In order to modularize the proof, we defined a lemma for each case.
As example, the following is the statement for modus ponens case.
\begin{minted}{coq}
Lemma Modus_Ponens_soundness:
  forall M w p q,
  (M ' w ||- p) /\ (M ' w ||- [! p -> q !]) ->
  (M ' w ||- q).
\end{minted}
All these lemmas are direct consequences from the definitions of the Kripke
semantics and entailment. We note that system K is used as it doesn't impose
any restrictions on the frame.

\paragraph{Frame Correspondence} 
As already mentioned, there are some modal formula schemas that correspond
to properties of the accessibility relation of a modal system, if we take them
as an axiom in the system.  While the proof that a certain property in a frame
implies the correspondent axiom is usually  direct, the inverse often needs
more effort. In the Coq proofs developed for this work,  the former case is
mostly unfoldings, simplifications and applications of premises
while the latter, apart from the necessity and burden of a proof by contraposition, 
requires the construction of a witness valuation function. 
Nevertheless, all Coq proofs are fairly similar to the respective
``pen and paper'' proofs found in the literature. 
Our work proves correspondence for the following properties: reflexivity,
transitivity, symmetry,  euclidean, seriality, functionality, density, convergency,
and converse well-foundness.

\section{Case Study}\label{sec:case}

In this section we describe a case study of the defined modal logic library:
a proof L\"ob theorem in an extension of system K4 with modal fixed points.
The importance of L\"ob's theorem is due to its use in provability
logic~\cite{boolos_1994}, used to abstract details from encodings in G\"odel's
incompleteness results.

In order to state L\"ob's theorem we need a way to encode modal
fixed points. Formally, let $\varphi(X)$ be a formula with only one propositional variable
$X$. The modal fixed point of $\varphi$ is then a formula $\Psi$ such that such that
$\Psi \leftrightarrow \varphi(\Box \Psi)$.
We represent the existence of modal fixed points for all one-variable formulas by a definition \mintinline{coq}{fixed_point},
which is parameterized by a system of modal logic, \mintinline{coq}{S}, and a
theory, \mintinline{coq}{g}, and encodes the logical equivalence demonstrating
that \mintinline{coq}{p} is a modal fixed point for \mintinline{coq}{f}.
\begin{minted}{coq}
  Definition fixed_point S g: Prop :=
    forall f, exists p,
    (S; g |-- [! (p <-> f ([]p)) !]).
\end{minted}

\noindent We can now write down L\"ob's theorem as follows:

\begin{minted}{coq}
  Theorem Lob:
    forall A, subset K4 A /\ fixed_point A nil ->
    forall P, (A; nil |-- [! []P -> P !]) ->
              (A; nil |-- [! P !]).
\end{minted}

\noindent The proof starts by assuming an arbitrary formula $P$, the existence of modal fixed
points and that $\Box P \to P$ is provable with axioms A. These deduction steps
are performed by the following tactic:

\begin{minted}{coq}
  intros A [I FP] P H1.
\end{minted}

\noindent Next, from the existence of modal fixed points for every formula,
we know that there is a formula $\Psi$ such that $\Psi \leftrightarrow (\Box
\Psi \to P)$. The following tactic performs this step of reasoning.

\begin{minted}{coq}
  destruct FP with (fun X => [! X -> P !]) as (psi, H2).
\end{minted}

\noindent Now, we are able to deduce that $\Psi \to (\Box \Psi\to P)$ from the previous
step result as follows:

\begin{minted}{coq}
  assert (A; nil |-- [! psi -> []psi -> P !]) as H3.
  apply modal_ax5 in H2; auto with modal.
\end{minted}

\noindent The \mintinline{coq}{assert} tactic allows the definition of
auxiliary proof goals. In the previous code piece, we specified
$\vdash \Psi \to (\Box \Psi \to P)$ as a new goal which
generates a new proof hypothesis named \mintinline{coq}{H3}. The
deduction of this fact is immediate from K4 axioms, of which A is a superset.
We use some basic automation to move on.

From $\vdash \Psi \to (\Box \Psi \to P)$ and the necessitation
rule, we can deduce $\vdash \Box (\Psi \to (\Box \Psi \to P))$
using the following tactics:
\begin{minted}{coq}
  assert (A; nil |-- [! [](psi -> []psi -> P) !])
    as H4.
  apply Nec; auto.
\end{minted}

\noindent Now, using the axiom \textbf{K} we can deduce $\vdash \Box \Psi \to
\Box (\Box \Psi \to P)$. Again, we deduce this fact using the
\mintinline{coq}{assert} tactic.

\begin{minted}{coq}
  assert (A; nil |-- [! []psi -> []([]psi -> P) !])
    as H5.
  apply modal_axK in H4; auto with modal.
\end{minted}

\noindent The next proof step involves showing that
$\vdash \Box (\Box \Psi \to P)\to (\Box \Box \Psi \to \Box P)$ holds
by axiom \textbf{K}.

\begin{minted}{coq}
  assert (A; nil |-- [! []([]psi -> P) ->
                          [][]psi -> []P !]) as H6.
  eapply Ax with (a := axK ?[X] ?[Y]); auto with modal.
  reflexivity.
\end{minted}

\noindent Using $\vdash \Box (\Box \Psi \to P)\to (\Box \Box \Psi \to \Box P)$ and
$\vdash \Box \Psi \to\Box (\Box \Psi \to P)$ we can deduce
$\vdash \Box \Psi \to \Box \Box \Psi \to \Box P$ using transitivity of
implication.

\begin{minted}{coq}
  assert (A; nil |-- [! []psi -> [][]psi -> []P !])
    as H7.
  eapply modal_compose; eauto with modal.
\end{minted}

\noindent Now, we can show that $\vdash\Box\Psi\to\Box\Box\Psi$ using
the axiom K4 of A. The next snippet of tactics perform this
reasoning step.

\begin{minted}{coq}
  assert (A; nil |-- [! []psi -> [][]psi !]) as H8.
  apply modal_axK4; auto with modal.
\end{minted}

\noindent From the previou two hypotheses we gather necessary info to derive
$\vdash \Box \Psi \to P$, using a propositional axiom.

\begin{minted}{coq}
  assert (A; nil |-- [! []psi -> []P !]) as H9.
  eapply modal_ax2; eauto with modal.
\end{minted}

\noindent Then, from $\vdash \Box\Psi \to \Box P$ and $\vdash \Box P \to P$, we can
conclude $\vdash \Box \Psi \to P$, by composition of implication as
follows.

\begin{minted}{coq}
  assert (A; nil |-- [! []psi -> P !]) as H10.
  eapply modal_compose; eauto with modal.
\end{minted}

\noindent Next, we can conclude $\vdash \Box (\Psi \to P) \to \Psi$ from
$\Psi \leftrightarrow (\Box \Psi \to P)$ using
the following tactics:

\begin{minted}{coq}
  assert (A; nil |-- [! ([]psi -> P) -> psi !]) as H11.
  apply modal_ax6 in H2; auto with modal.
\end{minted}

\noindent Now, from $\vdash (\Box\Psi \to P) \to \Psi$ and $\vdash \Box \Psi \to P$
we can conclude $\vdash \Psi$ by modus ponens.

\begin{minted}{coq}
  assert (A; nil |-- psi) as H12.
  eapply Mp; try eassumption.
\end{minted}

\noindent By necessitation rule, we can deduce $\vdash \Box \Psi$

\begin{minted}{coq}
  assert (A; nil |-- [! []psi !]) as H13.
  apply Nec; try eassumption.
\end{minted}

\noindent Finally, the desired conclusion $\vdash P$ follows from
using the rule modus ponens over $\vdash \Box \Psi \to P$
and $\vdash \Box \Psi$, as presented by the next tactics
snippet.

\begin{minted}{coq}
  eapply Mp; try eassumption.
\end{minted}

\noindent This concludes the formalization of L\"ob's theorem using
the developed library, in a Hilbert style proof.

\section{Related work}\label{sec:related}

\paragraph{Formalizations of modal logics in proof assistants}

The formalization of epistemic modal logic in Coq was the subject of de Wind's
work~\cite{dewind}.
The author made a Coq implementation of modal logic based on the S5 system,
which consists of  the axioms $K$, $T$, $4$ and $5$. The objective was the
modeling of two well known puzzles.
De Wind's formalization was separated into some components. The first consists in
defining the syntax of formulas and its Kripke semantics. Next, the author uses
their definitions to give some examples of modal operators, worlds and how they
interact with each other. Using the built machinery, the author developed
introduction and elimination rules for modal logic operators and use them to
model the wize man's and muddy children puzzles.

A Coq formalization for modal systems $K$, $T$ and $K4$ was the subject of 
Doczkal and Smolka's work. The authors' interest was in the definition of dynamic
and temporal logics. The development of the code establishes the construction
and definitions of tactics, through lemmas propositional operators and relation
properties were constructed. Each model was defined through structures, which
consist of worlds, relationships and propositions~\cite{saarland}.

Another proposal of formalizing modal logics was developed by Benzmüller and
Paleo~\cite{Chris_Bruno}. Author's main objective is checking if current
type theory-based proof assistants are suitable for modal logic reasoning.
They defined a formalization of systems $K$ and $S5$ and built custom tactics
which allows proofs using a natural deduction style.
The authors argue that current proof assistants are a useful technology for
reasoning about facts expressed using modal logics. 

The development of a completeness proof for modal logic S5 in Lean proof
assistant was the subject of the Bentzen's work~\cite{Bentzen2021}. The author
follows the approach due to Henkin for S5 completeness and it depends on
\texttt{mathlib}, the Lean's library of mathematical results~\cite{mathlib20}. 

Maggesi et al. developed a formalization of the completeness proof for GL using
the HOL light theorem prover~\cite{Maggesi21}. The authors argue that using the
standard Henkin strategy does not work in GL without modifications. The core idea of
Maggesi et al. strategy was building maximal consistent lists in terms of
components of the given formula non-proven in the calculus.

The following table summarizes the formalized modal logics systems by our library
and related works:
\vspace{0.5cm} {
  \begin{table}[H]
    \begin{tabular}{|c|c|c|c|c|c|c|c|c|c|}
      \hline
      \multirow{2}{*}{Authors} & \multicolumn{9}{c|}{Systems considered} \\ \cline{2-10} 
                               & K & D & B & T & K4 & K5 & S4 & S5 & GL\\ \hline
      Wind et al.~\cite{dewind}     & X &   &   & X & X  & X  & X  &  X & \\ \hline
      Doczkal et al.~\cite{saarland}   & X &   &   & X & X  &    &    &  &  \\ \hline
      Benzmüller et al.~\cite{Chris_Bruno}& X &   & X & X &  X &    &    & X &  \\ \hline
      Bentzen~\cite{Bentzen2021} & X & & & X & & X & X & X & \\ \hline
      Maggesi et al.~\cite{Maggesi21} & X & & & & & & & & X \\ \hline
      This work         & X & X & X & X & X  &  X &  X & X  & X \\ \hline
    \end{tabular}
    \centering
    \caption{Comparision with different Coq modal logics developements.}
    \label{tab:comparacao_trabalhos_rel}
  \end{table}
}

As we can see, as far we know, our work is the only formalization that covers systems
K, D, B, T, K4, K5, S4, S5 and GL. Also, since our deduction is
parameterized by system of axioms considered, it can be used to represent any
system of normal modal logics, unlike any other formalizations found in the literature.

\paragraph{Applications of modal logics in semantics} Another application of
modal logics is its use for reasoning about programming languages semantics.
Dynamic logic was introduced by Pratt in late 70's as modal logic for reasoning
about programs~\cite{Pratt76,Pratt80}. The key idea of dynamic logic is to express the
Hoare triples by a modality meaning ``after a program $p$ executes, formula
$\varphi$ will be true''. The programming languages considered in Pratt's
original work are composed by assignment and regular language operators
(union, concatenation and Kleene star)~\cite{Pratt76}.

An application of modal logic IS5 as a type system for a distributed
programming language was proposed in~\cite{Murphy04}. 
In that work, modalities are used to represent types for mobile code
and remote addresses. Murphy et al defined the language by interpreting
the possible worlds of the intutionistic modal logic IS5 to be nodes on a
network and the modalities $\Box$ and $\Diamond$ to reflect mobility and
locality, respectively. Authors develop an operational semantics and show
that their formalization is type-safe.

The application of a multimodal separation logic for the semantics of low-level languages
was the subject of Dockins et al.'s work~\cite{Dockins08}. In that work, authors
combine several modal operators that do not need to satisfy the same frame
conditions and seperation logics for reasoning about non-trivial subsets of
low-level programming languages. Authors developed their formalization through a
shallow embedding in Coq proof assistant and provide two case studies:
the verification of a low-level virtual machine and a C-like language.

A model for recursive, impredicatively quantified types with mutable references
was studied by Appel et al.~\cite{Appel07}. Their interest in formalizing such
advanced typing concepts was in the context of typed assembly languages.
The authors stabilished a semantic type soundness proof by defining a Kripke-style 
semantics of the G\"odel-L\"ob logic of provability. All their definitions were
also mechanized through a shallow embedding in Coq.

Finally, given that a deep embedding for the modal logic was used, it's
natural to wonder how that might interact with the program extraction feature. The
reader might refer to~\cite{Kavvos16c}, which presents a survey on the computational
interpretation of several modal logic systems. Whether or not our current implementation
is suitable for extraction is left for future work.

% The embedding of modal linear logic in Coq was the subject of
% Sadrzadeh's work~\cite{Soton03}. Like our work, Sadrzadeh's uses a
% deep-embedding approach and assumes that all modalities are reflexive
% and transitive. Using her embedding, Sadrzadeh shows some proofs of classical
% logic puzzles.

\section{Conclusion}\label{sec:conclusion}

In this work, we presented a sound deep embedding of arbitrary
normal modal logics in Coq. Our approach is able to represent deduction
uniformly over different modal logic systems by using different axiom sets
as a predicate parameter in the deduction inductive type. Using this generic
representation of deduction we are able to prove a soundness theorem, and, in
order to demonstrate the applicability of the proposed work, prove L\"ob's theorem
by assuming fixed points. % All source code in this article has
% been formalized in Coq version 8.15.0 and it is available at~\cite{modal-coq-lib}.

As future work we intend to prove the completeness result for the Hilbert
deduction system and continue the development of our library by including
support for multimodalities (fusion of modal logics) and non-normal modal
logic systems, along with improving usability through proof automation.

\begin{acks}
  The present work was supported by Fundação de Amparo à Pesquisa de Minas
  Gerais (FAPEMIG) under grant APQ-01683-21 and Fundação de Amparo à Pesquisa de
  Santa Catarina (FAPESC). %under grant 2021TR961.
\end{acks}


\bibliographystyle{ACM-Reference-Format}
\bibliography{references}

\end{document}
